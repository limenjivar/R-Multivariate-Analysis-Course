\documentclass[]{article}
\usepackage{lmodern}
\usepackage{amssymb,amsmath}
\usepackage{ifxetex,ifluatex}
\usepackage{fixltx2e} % provides \textsubscript
\ifnum 0\ifxetex 1\fi\ifluatex 1\fi=0 % if pdftex
  \usepackage[T1]{fontenc}
  \usepackage[utf8]{inputenc}
\else % if luatex or xelatex
  \ifxetex
    \usepackage{mathspec}
  \else
    \usepackage{fontspec}
  \fi
  \defaultfontfeatures{Ligatures=TeX,Scale=MatchLowercase}
\fi
% use upquote if available, for straight quotes in verbatim environments
\IfFileExists{upquote.sty}{\usepackage{upquote}}{}
% use microtype if available
\IfFileExists{microtype.sty}{%
\usepackage[]{microtype}
\UseMicrotypeSet[protrusion]{basicmath} % disable protrusion for tt fonts
}{}
\PassOptionsToPackage{hyphens}{url} % url is loaded by hyperref
\usepackage[unicode=true]{hyperref}
\hypersetup{
            pdfborder={0 0 0},
            breaklinks=true}
\urlstyle{same}  % don't use monospace font for urls
\usepackage[margin=1in]{geometry}
\usepackage{color}
\usepackage{fancyvrb}
\newcommand{\VerbBar}{|}
\newcommand{\VERB}{\Verb[commandchars=\\\{\}]}
\DefineVerbatimEnvironment{Highlighting}{Verbatim}{commandchars=\\\{\}}
% Add ',fontsize=\small' for more characters per line
\usepackage{framed}
\definecolor{shadecolor}{RGB}{248,248,248}
\newenvironment{Shaded}{\begin{snugshade}}{\end{snugshade}}
\newcommand{\KeywordTok}[1]{\textcolor[rgb]{0.13,0.29,0.53}{\textbf{#1}}}
\newcommand{\DataTypeTok}[1]{\textcolor[rgb]{0.13,0.29,0.53}{#1}}
\newcommand{\DecValTok}[1]{\textcolor[rgb]{0.00,0.00,0.81}{#1}}
\newcommand{\BaseNTok}[1]{\textcolor[rgb]{0.00,0.00,0.81}{#1}}
\newcommand{\FloatTok}[1]{\textcolor[rgb]{0.00,0.00,0.81}{#1}}
\newcommand{\ConstantTok}[1]{\textcolor[rgb]{0.00,0.00,0.00}{#1}}
\newcommand{\CharTok}[1]{\textcolor[rgb]{0.31,0.60,0.02}{#1}}
\newcommand{\SpecialCharTok}[1]{\textcolor[rgb]{0.00,0.00,0.00}{#1}}
\newcommand{\StringTok}[1]{\textcolor[rgb]{0.31,0.60,0.02}{#1}}
\newcommand{\VerbatimStringTok}[1]{\textcolor[rgb]{0.31,0.60,0.02}{#1}}
\newcommand{\SpecialStringTok}[1]{\textcolor[rgb]{0.31,0.60,0.02}{#1}}
\newcommand{\ImportTok}[1]{#1}
\newcommand{\CommentTok}[1]{\textcolor[rgb]{0.56,0.35,0.01}{\textit{#1}}}
\newcommand{\DocumentationTok}[1]{\textcolor[rgb]{0.56,0.35,0.01}{\textbf{\textit{#1}}}}
\newcommand{\AnnotationTok}[1]{\textcolor[rgb]{0.56,0.35,0.01}{\textbf{\textit{#1}}}}
\newcommand{\CommentVarTok}[1]{\textcolor[rgb]{0.56,0.35,0.01}{\textbf{\textit{#1}}}}
\newcommand{\OtherTok}[1]{\textcolor[rgb]{0.56,0.35,0.01}{#1}}
\newcommand{\FunctionTok}[1]{\textcolor[rgb]{0.00,0.00,0.00}{#1}}
\newcommand{\VariableTok}[1]{\textcolor[rgb]{0.00,0.00,0.00}{#1}}
\newcommand{\ControlFlowTok}[1]{\textcolor[rgb]{0.13,0.29,0.53}{\textbf{#1}}}
\newcommand{\OperatorTok}[1]{\textcolor[rgb]{0.81,0.36,0.00}{\textbf{#1}}}
\newcommand{\BuiltInTok}[1]{#1}
\newcommand{\ExtensionTok}[1]{#1}
\newcommand{\PreprocessorTok}[1]{\textcolor[rgb]{0.56,0.35,0.01}{\textit{#1}}}
\newcommand{\AttributeTok}[1]{\textcolor[rgb]{0.77,0.63,0.00}{#1}}
\newcommand{\RegionMarkerTok}[1]{#1}
\newcommand{\InformationTok}[1]{\textcolor[rgb]{0.56,0.35,0.01}{\textbf{\textit{#1}}}}
\newcommand{\WarningTok}[1]{\textcolor[rgb]{0.56,0.35,0.01}{\textbf{\textit{#1}}}}
\newcommand{\AlertTok}[1]{\textcolor[rgb]{0.94,0.16,0.16}{#1}}
\newcommand{\ErrorTok}[1]{\textcolor[rgb]{0.64,0.00,0.00}{\textbf{#1}}}
\newcommand{\NormalTok}[1]{#1}
\usepackage{graphicx,grffile}
\makeatletter
\def\maxwidth{\ifdim\Gin@nat@width>\linewidth\linewidth\else\Gin@nat@width\fi}
\def\maxheight{\ifdim\Gin@nat@height>\textheight\textheight\else\Gin@nat@height\fi}
\makeatother
% Scale images if necessary, so that they will not overflow the page
% margins by default, and it is still possible to overwrite the defaults
% using explicit options in \includegraphics[width, height, ...]{}
\setkeys{Gin}{width=\maxwidth,height=\maxheight,keepaspectratio}
\IfFileExists{parskip.sty}{%
\usepackage{parskip}
}{% else
\setlength{\parindent}{0pt}
\setlength{\parskip}{6pt plus 2pt minus 1pt}
}
\setlength{\emergencystretch}{3em}  % prevent overfull lines
\providecommand{\tightlist}{%
  \setlength{\itemsep}{0pt}\setlength{\parskip}{0pt}}
\setcounter{secnumdepth}{0}
% Redefines (sub)paragraphs to behave more like sections
\ifx\paragraph\undefined\else
\let\oldparagraph\paragraph
\renewcommand{\paragraph}[1]{\oldparagraph{#1}\mbox{}}
\fi
\ifx\subparagraph\undefined\else
\let\oldsubparagraph\subparagraph
\renewcommand{\subparagraph}[1]{\oldsubparagraph{#1}\mbox{}}
\fi

% set default figure placement to htbp
\makeatletter
\def\fps@figure{htbp}
\makeatother


\author{}
\date{\vspace{-2.5em}}

\begin{document}

\section{Exam 2 Chapters 4 and 5}\label{exam-2-chapters-4-and-5}

\subsection{by Ledin Menjivar}\label{by-ledin-menjivar}

\subsubsection{Task 1}\label{task-1}

\textbf{1.}

\begin{Shaded}
\begin{Highlighting}[]
\KeywordTok{head}\NormalTok{(iris)}
\end{Highlighting}
\end{Shaded}

\begin{verbatim}
##   Sepal.Length Sepal.Width Petal.Length Petal.Width Species
## 1          5.1         3.5          1.4         0.2  setosa
## 2          4.9         3.0          1.4         0.2  setosa
## 3          4.7         3.2          1.3         0.2  setosa
## 4          4.6         3.1          1.5         0.2  setosa
## 5          5.0         3.6          1.4         0.2  setosa
## 6          5.4         3.9          1.7         0.4  setosa
\end{verbatim}

\begin{Shaded}
\begin{Highlighting}[]
\KeywordTok{str}\NormalTok{(iris)}
\end{Highlighting}
\end{Shaded}

\begin{verbatim}
## 'data.frame':    150 obs. of  5 variables:
##  $ Sepal.Length: num  5.1 4.9 4.7 4.6 5 5.4 4.6 5 4.4 4.9 ...
##  $ Sepal.Width : num  3.5 3 3.2 3.1 3.6 3.9 3.4 3.4 2.9 3.1 ...
##  $ Petal.Length: num  1.4 1.4 1.3 1.5 1.4 1.7 1.4 1.5 1.4 1.5 ...
##  $ Petal.Width : num  0.2 0.2 0.2 0.2 0.2 0.4 0.3 0.2 0.2 0.1 ...
##  $ Species     : Factor w/ 3 levels "setosa","versicolor",..: 1 1 1 1 1 1 1 1 1 1 ...
\end{verbatim}

\begin{Shaded}
\begin{Highlighting}[]
\KeywordTok{summary}\NormalTok{(iris)}
\end{Highlighting}
\end{Shaded}

\begin{verbatim}
##   Sepal.Length    Sepal.Width     Petal.Length    Petal.Width   
##  Min.   :4.300   Min.   :2.000   Min.   :1.000   Min.   :0.100  
##  1st Qu.:5.100   1st Qu.:2.800   1st Qu.:1.600   1st Qu.:0.300  
##  Median :5.800   Median :3.000   Median :4.350   Median :1.300  
##  Mean   :5.843   Mean   :3.057   Mean   :3.758   Mean   :1.199  
##  3rd Qu.:6.400   3rd Qu.:3.300   3rd Qu.:5.100   3rd Qu.:1.800  
##  Max.   :7.900   Max.   :4.400   Max.   :6.900   Max.   :2.500  
##        Species  
##  setosa    :50  
##  versicolor:50  
##  virginica :50  
##                 
##                 
## 
\end{verbatim}

\textbf{2.} The dataset iris contains 150 observations for 3 different
flowers setosa, versicolor and virginica. And it contains the
measurements for sepal length and width and the measurements for petal
length and width. These measurements are in centimeters for the
different species of flowers. The help() function was also used.

\textbf{3.} The head() function will by default return the first 6
observations within our dataset.The number of obeservations can also be
specified with heaad(iris,n) where n is the number of observations to be
returned.

\textbf{4}

\begin{Shaded}
\begin{Highlighting}[]
\KeywordTok{boxplot}\NormalTok{(}\DataTypeTok{formula=}\NormalTok{Petal.Length}\OperatorTok{~}\NormalTok{Species, }\DataTypeTok{data =}\NormalTok{ iris)}
\end{Highlighting}
\end{Shaded}

\includegraphics{Exam2_files/figure-latex/unnamed-chunk-2-1.pdf}

\begin{Shaded}
\begin{Highlighting}[]
\KeywordTok{boxplot}\NormalTok{(}\DataTypeTok{formula=}\NormalTok{Petal.Length}\OperatorTok{~}\NormalTok{Species, }\DataTypeTok{data =}\NormalTok{ iris, }\DataTypeTok{plot=}\NormalTok{F)}
\end{Highlighting}
\end{Shaded}

\begin{verbatim}
## $stats
##      [,1] [,2] [,3]
## [1,]  1.1 3.30 4.50
## [2,]  1.4 4.00 5.10
## [3,]  1.5 4.35 5.55
## [4,]  1.6 4.60 5.90
## [5,]  1.9 5.10 6.90
## 
## $n
## [1] 50 50 50
## 
## $conf
##          [,1]     [,2]     [,3]
## [1,] 1.455311 4.215933 5.371243
## [2,] 1.544689 4.484067 5.728757
## 
## $out
## [1] 1 3
## 
## $group
## [1] 1 2
## 
## $names
## [1] "setosa"     "versicolor" "virginica"
\end{verbatim}

The assigned code for this problem returns creates a boxplot which shows
petal lengths for each species of flower in the dataset iris. By setting
\(plot=F\) we can also see the actual statistical values used to create
each boxplot. From the boxplot figure we can see we have two outliers, 1
in setosa species and another in the versicolor species of flowers. We
can also see that we have the medians: \(setosa= 1.5cm\),
\(versicolor=4.35cm\) and \(virginica=5.55cm\). Thus we can determine
that the species virginica will have a longer petal length on average.
From our diagram we can also see that for the setosa species we have
smaller upper and lower extremums than the other two flowers which we
have lower extremum for setosa \(= 1.1cm\) and upper extremum for setosa
= \(1.9cm\). Thus we can also determine that the setosa species will
always have smaller petal lengths than that of the other two species. We
can also see that we may sometimes have longer petal lengths for the
species versicolor than that of the virginica flower because of: upper
extremum of versicolor = \(5.10cm\) and we have lower extremum of
virginica\(=4.50\).

\textbf{5.}

\begin{Shaded}
\begin{Highlighting}[]
\NormalTok{model1 <-}\StringTok{ }\KeywordTok{manova}\NormalTok{(}\KeywordTok{cbind}\NormalTok{(Sepal.Length,Sepal.Width) }\OperatorTok{~}\StringTok{ }\NormalTok{Species, }\DataTypeTok{data=}\NormalTok{iris)}
\end{Highlighting}
\end{Shaded}

\textbf{6.} The code chunk in question \(5\) tests for variance when
there are more than one dependent variables. Here our dependent
variables are Sepal.Length and Sepal.Width. And our independent variable
is Species.

\textbf{7.} No. Because there is no print or summary statement to show
the results.

\textbf{8.}

\begin{Shaded}
\begin{Highlighting}[]
\KeywordTok{summary.aov}\NormalTok{(model1)}
\end{Highlighting}
\end{Shaded}

\begin{verbatim}
##  Response Sepal.Length :
##              Df Sum Sq Mean Sq F value    Pr(>F)    
## Species       2 63.212  31.606  119.26 < 2.2e-16 ***
## Residuals   147 38.956   0.265                      
## ---
## Signif. codes:  0 '***' 0.001 '**' 0.01 '*' 0.05 '.' 0.1 ' ' 1
## 
##  Response Sepal.Width :
##              Df Sum Sq Mean Sq F value    Pr(>F)    
## Species       2 11.345  5.6725   49.16 < 2.2e-16 ***
## Residuals   147 16.962  0.1154                      
## ---
## Signif. codes:  0 '***' 0.001 '**' 0.01 '*' 0.05 '.' 0.1 ' ' 1
\end{verbatim}

\textbf{9.} We have a p-value of \(2.2e^{-16}\) for both Sepal.Length
and Sepal.Width. Thus we have a statistically significant value. Hence
we can conclude that there is no variability betwee the different
species of flowers.

\subsubsection{Task 2}\label{task-2}

\textbf{1}

\begin{Shaded}
\begin{Highlighting}[]
\KeywordTok{data}\NormalTok{(iris)}
\NormalTok{iris}
\end{Highlighting}
\end{Shaded}

\begin{verbatim}
##     Sepal.Length Sepal.Width Petal.Length Petal.Width    Species
## 1            5.1         3.5          1.4         0.2     setosa
## 2            4.9         3.0          1.4         0.2     setosa
## 3            4.7         3.2          1.3         0.2     setosa
## 4            4.6         3.1          1.5         0.2     setosa
## 5            5.0         3.6          1.4         0.2     setosa
## 6            5.4         3.9          1.7         0.4     setosa
## 7            4.6         3.4          1.4         0.3     setosa
## 8            5.0         3.4          1.5         0.2     setosa
## 9            4.4         2.9          1.4         0.2     setosa
## 10           4.9         3.1          1.5         0.1     setosa
## 11           5.4         3.7          1.5         0.2     setosa
## 12           4.8         3.4          1.6         0.2     setosa
## 13           4.8         3.0          1.4         0.1     setosa
## 14           4.3         3.0          1.1         0.1     setosa
## 15           5.8         4.0          1.2         0.2     setosa
## 16           5.7         4.4          1.5         0.4     setosa
## 17           5.4         3.9          1.3         0.4     setosa
## 18           5.1         3.5          1.4         0.3     setosa
## 19           5.7         3.8          1.7         0.3     setosa
## 20           5.1         3.8          1.5         0.3     setosa
## 21           5.4         3.4          1.7         0.2     setosa
## 22           5.1         3.7          1.5         0.4     setosa
## 23           4.6         3.6          1.0         0.2     setosa
## 24           5.1         3.3          1.7         0.5     setosa
## 25           4.8         3.4          1.9         0.2     setosa
## 26           5.0         3.0          1.6         0.2     setosa
## 27           5.0         3.4          1.6         0.4     setosa
## 28           5.2         3.5          1.5         0.2     setosa
## 29           5.2         3.4          1.4         0.2     setosa
## 30           4.7         3.2          1.6         0.2     setosa
## 31           4.8         3.1          1.6         0.2     setosa
## 32           5.4         3.4          1.5         0.4     setosa
## 33           5.2         4.1          1.5         0.1     setosa
## 34           5.5         4.2          1.4         0.2     setosa
## 35           4.9         3.1          1.5         0.2     setosa
## 36           5.0         3.2          1.2         0.2     setosa
## 37           5.5         3.5          1.3         0.2     setosa
## 38           4.9         3.6          1.4         0.1     setosa
## 39           4.4         3.0          1.3         0.2     setosa
## 40           5.1         3.4          1.5         0.2     setosa
## 41           5.0         3.5          1.3         0.3     setosa
## 42           4.5         2.3          1.3         0.3     setosa
## 43           4.4         3.2          1.3         0.2     setosa
## 44           5.0         3.5          1.6         0.6     setosa
## 45           5.1         3.8          1.9         0.4     setosa
## 46           4.8         3.0          1.4         0.3     setosa
## 47           5.1         3.8          1.6         0.2     setosa
## 48           4.6         3.2          1.4         0.2     setosa
## 49           5.3         3.7          1.5         0.2     setosa
## 50           5.0         3.3          1.4         0.2     setosa
## 51           7.0         3.2          4.7         1.4 versicolor
## 52           6.4         3.2          4.5         1.5 versicolor
## 53           6.9         3.1          4.9         1.5 versicolor
## 54           5.5         2.3          4.0         1.3 versicolor
## 55           6.5         2.8          4.6         1.5 versicolor
## 56           5.7         2.8          4.5         1.3 versicolor
## 57           6.3         3.3          4.7         1.6 versicolor
## 58           4.9         2.4          3.3         1.0 versicolor
## 59           6.6         2.9          4.6         1.3 versicolor
## 60           5.2         2.7          3.9         1.4 versicolor
## 61           5.0         2.0          3.5         1.0 versicolor
## 62           5.9         3.0          4.2         1.5 versicolor
## 63           6.0         2.2          4.0         1.0 versicolor
## 64           6.1         2.9          4.7         1.4 versicolor
## 65           5.6         2.9          3.6         1.3 versicolor
## 66           6.7         3.1          4.4         1.4 versicolor
## 67           5.6         3.0          4.5         1.5 versicolor
## 68           5.8         2.7          4.1         1.0 versicolor
## 69           6.2         2.2          4.5         1.5 versicolor
## 70           5.6         2.5          3.9         1.1 versicolor
## 71           5.9         3.2          4.8         1.8 versicolor
## 72           6.1         2.8          4.0         1.3 versicolor
## 73           6.3         2.5          4.9         1.5 versicolor
## 74           6.1         2.8          4.7         1.2 versicolor
## 75           6.4         2.9          4.3         1.3 versicolor
## 76           6.6         3.0          4.4         1.4 versicolor
## 77           6.8         2.8          4.8         1.4 versicolor
## 78           6.7         3.0          5.0         1.7 versicolor
## 79           6.0         2.9          4.5         1.5 versicolor
## 80           5.7         2.6          3.5         1.0 versicolor
## 81           5.5         2.4          3.8         1.1 versicolor
## 82           5.5         2.4          3.7         1.0 versicolor
## 83           5.8         2.7          3.9         1.2 versicolor
## 84           6.0         2.7          5.1         1.6 versicolor
## 85           5.4         3.0          4.5         1.5 versicolor
## 86           6.0         3.4          4.5         1.6 versicolor
## 87           6.7         3.1          4.7         1.5 versicolor
## 88           6.3         2.3          4.4         1.3 versicolor
## 89           5.6         3.0          4.1         1.3 versicolor
## 90           5.5         2.5          4.0         1.3 versicolor
## 91           5.5         2.6          4.4         1.2 versicolor
## 92           6.1         3.0          4.6         1.4 versicolor
## 93           5.8         2.6          4.0         1.2 versicolor
## 94           5.0         2.3          3.3         1.0 versicolor
## 95           5.6         2.7          4.2         1.3 versicolor
## 96           5.7         3.0          4.2         1.2 versicolor
## 97           5.7         2.9          4.2         1.3 versicolor
## 98           6.2         2.9          4.3         1.3 versicolor
## 99           5.1         2.5          3.0         1.1 versicolor
## 100          5.7         2.8          4.1         1.3 versicolor
## 101          6.3         3.3          6.0         2.5  virginica
## 102          5.8         2.7          5.1         1.9  virginica
## 103          7.1         3.0          5.9         2.1  virginica
## 104          6.3         2.9          5.6         1.8  virginica
## 105          6.5         3.0          5.8         2.2  virginica
## 106          7.6         3.0          6.6         2.1  virginica
## 107          4.9         2.5          4.5         1.7  virginica
## 108          7.3         2.9          6.3         1.8  virginica
## 109          6.7         2.5          5.8         1.8  virginica
## 110          7.2         3.6          6.1         2.5  virginica
## 111          6.5         3.2          5.1         2.0  virginica
## 112          6.4         2.7          5.3         1.9  virginica
## 113          6.8         3.0          5.5         2.1  virginica
## 114          5.7         2.5          5.0         2.0  virginica
## 115          5.8         2.8          5.1         2.4  virginica
## 116          6.4         3.2          5.3         2.3  virginica
## 117          6.5         3.0          5.5         1.8  virginica
## 118          7.7         3.8          6.7         2.2  virginica
## 119          7.7         2.6          6.9         2.3  virginica
## 120          6.0         2.2          5.0         1.5  virginica
## 121          6.9         3.2          5.7         2.3  virginica
## 122          5.6         2.8          4.9         2.0  virginica
## 123          7.7         2.8          6.7         2.0  virginica
## 124          6.3         2.7          4.9         1.8  virginica
## 125          6.7         3.3          5.7         2.1  virginica
## 126          7.2         3.2          6.0         1.8  virginica
## 127          6.2         2.8          4.8         1.8  virginica
## 128          6.1         3.0          4.9         1.8  virginica
## 129          6.4         2.8          5.6         2.1  virginica
## 130          7.2         3.0          5.8         1.6  virginica
## 131          7.4         2.8          6.1         1.9  virginica
## 132          7.9         3.8          6.4         2.0  virginica
## 133          6.4         2.8          5.6         2.2  virginica
## 134          6.3         2.8          5.1         1.5  virginica
## 135          6.1         2.6          5.6         1.4  virginica
## 136          7.7         3.0          6.1         2.3  virginica
## 137          6.3         3.4          5.6         2.4  virginica
## 138          6.4         3.1          5.5         1.8  virginica
## 139          6.0         3.0          4.8         1.8  virginica
## 140          6.9         3.1          5.4         2.1  virginica
## 141          6.7         3.1          5.6         2.4  virginica
## 142          6.9         3.1          5.1         2.3  virginica
## 143          5.8         2.7          5.1         1.9  virginica
## 144          6.8         3.2          5.9         2.3  virginica
## 145          6.7         3.3          5.7         2.5  virginica
## 146          6.7         3.0          5.2         2.3  virginica
## 147          6.3         2.5          5.0         1.9  virginica
## 148          6.5         3.0          5.2         2.0  virginica
## 149          6.2         3.4          5.4         2.3  virginica
## 150          5.9         3.0          5.1         1.8  virginica
\end{verbatim}

\begin{Shaded}
\begin{Highlighting}[]
\CommentTok{#install.packages('jmv')}
\KeywordTok{library}\NormalTok{(jmv)}
\end{Highlighting}
\end{Shaded}

\begin{verbatim}
## Warning: package 'jmv' was built under R version 4.0.4
\end{verbatim}

\begin{Shaded}
\begin{Highlighting}[]
\KeywordTok{mancova}\NormalTok{(}\DataTypeTok{data=}\NormalTok{iris,}\DataTypeTok{deps=}\KeywordTok{vars}\NormalTok{(Sepal.Length,Sepal.Width,Petal.Length,Petal.Width),}\DataTypeTok{factor=}\NormalTok{Species)}
\end{Highlighting}
\end{Shaded}

\begin{verbatim}
## 
##  MANCOVA
## 
##  Multivariate Tests                                                                      
##  --------------------------------------------------------------------------------------- 
##                                     value         F           df1    df2    p            
##  --------------------------------------------------------------------------------------- 
##    Species    Pillai's Trace          1.191899    53.46649      8    290    < .0000001   
##               Wilks' Lambda         0.02343863    199.1453      8    288    < .0000001   
##               Hotelling's Trace       32.47732    580.5321      8    286    < .0000001   
##               Roy's Largest Root      32.19193    1166.957      4    145    < .0000001   
##  --------------------------------------------------------------------------------------- 
## 
## 
##  Univariate Tests                                                                                         
##  -------------------------------------------------------------------------------------------------------- 
##                 Dependent Variable    Sum of Squares    df     Mean Square     F             p            
##  -------------------------------------------------------------------------------------------------------- 
##    Species      Sepal.Length               63.212133      2     31.60606667     119.26450    < .0000001   
##                 Sepal.Width                11.344933      2      5.67246667      49.16004    < .0000001   
##                 Petal.Length              437.102800      2    218.55140000    1180.16118    < .0000001   
##                 Petal.Width                80.413333      2     40.20666667     960.00715    < .0000001   
##    Residuals    Sepal.Length               38.956200    147      0.26500816                               
##                 Sepal.Width                16.962000    147      0.11538776                               
##                 Petal.Length               27.222600    147      0.18518776                               
##                 Petal.Width                 6.156600    147      0.04188163                               
##  --------------------------------------------------------------------------------------------------------
\end{verbatim}

\end{document}
